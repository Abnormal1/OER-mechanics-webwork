\documentclass[a4paper]{article}

\usepackage{microtype}
\usepackage{graphicx}
\usepackage{enumitem}
\usepackage{amsmath}

\begin{document}
\title{Mechanics Map Formatting Guidelines}

\begin{center} \textbf{Mechanics Map Formatting Guidelines}
\end{center}

\underline{Terminology:}
\begin{itemize}
	\item Engineering Mechanics - Collectively, the study of the interaction of forces, bodies, and motion
	\item Statics - The study of rigid bodies in equilibrium
	\item Dynamics - The study of rigid bodies in motion
	\item Strength of Materials - The study of deformable bodies
	\item Particles - A body where we assume all mass is concentrated at a single point.
	\begin{itemize}
				\item Alternatively we will talk about concurrent force systems, which we will approximate as particles
	\end{itemize}
	\item Rigid Bodies - A body that is assumed to not deform under loading and that has a distributed mass
\end{itemize}

\underline{Website Format:}
\begin{itemize}
	\item Webpage
		\begin{itemize}
				\item Each html page should have a title "Mechanics Map - *Subject Title*"
				\item Use the \textless h1\textgreater \space tags for a visible title on the top of the page
				\item Use the \textless h2\textgreater \space tags for section headings within the page
				\item Use the \textless strong\textgreater \space tags to bold important terms used for the first time
		\end{itemize}
	\item Images
		\begin{itemize}
				\item Use public domain images or self-generated images if possible
				\item Images under a CC-BY-SA or CC-BY can also be used, with the source being attributed in the image caption
				\item Always include an image caption
				\item Images in the main content area should be no more than 600px in width (standard should be 500px)
				\item Worked problem images should be no more than 500px in width
		\end{itemize}
	\item Equations
		\begin{itemize}
				\item For accessibility reasons, all equatins should be written in LaTeX using the MathJax pluggin
					\begin{itemize}
						\item Use an equation table to organize and center the equation
						\item Write the equation itself in the MathJax tags
						\item \textbackslash[ *Put LaTeX equation here* \textbackslash]
					\end {itemize}
		\end{itemize}
\end {itemize}

 \pagebreak
\underline{Symbols:}
\begin{itemize}
	\item Body Physical Properties and Points
		\begin{itemize}
			\item \(m\): mass
			\item \(C\): Centroid point for a 2D area
			\item \( \bar{x}\) and \( \bar{y} \) for the x- and y-coordinates of the centroid
			\item \(G\): Center of mass point
			\item \(O\): A fixed ground point, particularly for fixed axis rotation
			\item Other points should generally be labelled \(A, B, C, etc.\)
		\end{itemize}
	\item Vectors
		\begin{itemize}
			\item Vectors use a rightward arrow over the variable (the \textbackslash vec\{\} tag in LaTeX)
		\end{itemize}
	\item Forces
		\begin{itemize}
			\item \(F\): A force
			\item \(F_A\): A force at point A
			\item \(F_{Ax} \): The x component of the force at A
			\item \(F_g\): The gravity force
			\item \(F_N\): A normal force
			\item \(T\): A tension force
			\item \(F_k\): Force from a spring
			\item \(F_c\): Force from a damper
		\end{itemize}
	\item Moments
		\begin{itemize}
			\item \(M\): A moment
			\item \(M_A\): The moment about point A
			\item \(M_{Ax}\): The moment about point A about the x-axis
		\end{itemize}
	\item Moments of Inertia
		\begin{itemize}
			\item \(I_G\): Mass moments of inertia for 2D problems, always use subscript to denote point the moment of inertia is about
			\item \(I_{xx}\), \(I_{yy}\), and \(I_{zz}\) for mass moments of inertia in 3D about the center of mass
			\item \(I_{xx0}\), \(I_{yy0}\), and \(I_{zz0}\) for mass moments of inertia in 3D about a point other than the center of mass
			\item \(K\) is used for the radius of gyration
		\end{itemize}
	\item Motion in one dimension
		\begin{itemize}
			\item \(x\): position in one dimension
			\item \(\dot{x}\) or \(v\): velocity in one dimension
			\item \(\ddot{x}\) or \(a\): acceleration in one dimension
		\end{itemize}
	\item Motion in x-y coordinates
		\begin{itemize}
			\item \(x\) and \(y\) position
			\item \(\dot{x}\) or \(v_x\) and \(\dot{y}\) or \(v_y\)
			\item \(\ddot{x}\) or \(a_x\) and \(\ddot{y}\) or \(a_y\) accelerations
		\end{itemize}
	\item Motion in n-t coordinates
		\begin{itemize}
			\item \(v_t\): velocity
			\item \(a_t\) or \(\dot{v}\) and \(a_N\) accelerations
		\end{itemize}
	\item Motion in polar coordinates
		\begin{itemize}
			\item \(r\) and \(\theta\) for position
			\item \(v_r\) and \(v_\theta\) for velocities
			\item \(a_r\) and \(a_\theta\) for acceleration
		\end{itemize}
	\item Relative Motion
		\begin{itemize}
			\item \(\vec{r}_{A/B}\): Position of A with respect to B
			\item \(\vec{r}_{A/O}\): Position of A with respect to the origin
			\item \(\vec{v}_{A/B}\): Velocity of A with respect to B
			\item \(\vec{v}_{A/O}\): Velocity of A with respect to the origin
			\item \(\vec{a}_{A/B}\): Acceleration of A with respect to B
			\item \(\vec{a}_{A/O}\): Acceleration of A with respect to the origin
		\end{itemize}
	\item Coordinate systems for Relative Motion Analysis
		\begin{itemize}
			\item x- and y-coordinates reserved for the fixed ground frame of reference
			\item Use \(r_1\) and \(\theta_1\) coordinate directions for first coordinate system that rotates with the body (mirroring polar kinematics), increase the subscript number for each additional rotating coordinate system that is required
		\end{itemize}
	\item Work and Energy
		\begin{itemize}
			\item W: Work
			\item KE: Kinetic energy
			\item PE: Potential energy
			\item P: Power
			\item \(\eta\): Efficiency
		\end{itemize}
	\item Impulse and Momentum
		\begin{itemize}
			\item \(\vec{J}\): Linear impulse (use vector in for vector form, use subscripts (\(J_x\)) when discussing components
			\item \(m\vec{v}\): Momentum (use vector when in vector form, use subscripts for initial and final, and for directions when breaking it down into components)
			\item \(mv_{Afx}\): For 2D collisions, use subscripts (in this order) to describe the body, initial vs. final, and the direction
			\item \(\vec{K}\): Angular impulse (use vector in for vector form, use subscripts when discussing components
		\end{itemize}
	\item Vibrations
		\begin{itemize}
			\item \(k\): Spring constant
			\item \(k_{eq}\): Equivalent spring constant
			\item \(c\): Damping constant
			\item \(\omega_0\): Forced frequency
			\item \(\omega_n\): Natural frequency
			\item \(\omega_d\): Damped natural frequency
		\end{itemize}
\end{itemize}

\underline{Graphics:}
	\begin{itemize}
		\item Free Body Diagrams
			\begin{itemize}
				\item Free body diagrams should show only the body (no background), with the body itself in black
				\item Coordinate systems should be drawn as appropriate in black
				\item The coordinate system should be drawn on the diagram
				\item Forces should be shown as red vectors
				\item Moments in planar problems should be shown as purple curving vectors
				\item If velocities or accelerations are shown, use a dashed blue vector
				\item Key dimensions should be shown in blue
			\end{itemize}
	\end{itemize}

\underline{Video Formatting}
	\begin{itemize}
		\item Videos
			\begin{itemize}
				\item Videos should be uploaded to the group YouTube account
				\item Each video lecture should be titled "*Section Number* *Subject Name* - Video Lecture - *Your initials*"
				\item Each worked problem should be titled "*Section Number* *Subject Name* - WP\#\#\# - *Your initials*"
				\item Please have the problem itself shown at the beginning of the video and give a brief verbal recap of the problem
				\item Don't refer to the problem number in the video itself (just say "In this problem..."). This makes iteasier if we add problems and change numbers
		\end{itemize}
	\end{itemize}


\end{document}

